\documentclass[uplatex,paper=a4,fontsize=4.0truemm,jafontsize=4.0truemm,head_space=30.0truemm,foot_space=30.0truemm,baselineskip=8.0truemm,line_length=40zw,gutter=25.0truemm,oneside,openany,fleqn,hanging_panctuation,open_bracket_pos=nibu_tentsuki,dvipdfmx,jis2004,book,titlepage]{jlreq}
%
\西暦
%\和暦
\NewPageStyle{TopRightNombre}{nombre_position=top-right}
\pagestyle{TopRightNombre}
\jlreqsetup{%
	caption_font={\small\sffamily},%
	caption_label_font={\small\sffamily}%
}
\ModifyHeading{chapter}{lines=3}
\ModifyHeading{section}{lines=2}
\ModifyHeading{subsection}{%
	lines=1,
	font={\large\sffamily}%
}
\usepackage{amsmath,amssymb}
\usepackage[OT1,T2A,X2,OT2,LGR,T1]{fontenc}
\usepackage{textcomp}
\usepackage{lmodern,exscale}
%\usepackage{newtxtext,newtxmath}
%\usepackage{newpxtext,newpxmath}
\usepackage{mathrsfs,bm,myFonts}%myFonts:自作パッケージ@D:\texlive\texmf-local\tex\latex\local\local\myFonts
\usepackage[expert,deluxe]{otf}
\usepackage[noalphabet,unicode,sourcehan-otc]{pxchfon}
\usepackage[prefernoncjk]{pxcjkcat}
\cjkcategory{sym04,sym08,sym09,sym10,sym11,sym12,sym15,sym18,sym19,sym20,brai,sym33}{cjk}
\usepackage[polutonikogreek,esperanto,serbian,bulgarian,ukrainian,russian,greek,french,italian,german,ngerman,dutch,tchinese,schinese,korean,english,japanese]{pxbabel}
\usepackage{soul,dotlessj,myt2aenc,myx2enc,bxokumacro,pxrubrica}
\usepackage{zi4,bxinconsolata,upquote}
\usepackage{mathtools,fouridx}
\usepackage{bxascmac}
\usepackage{tcolorbox}
\tcbuselibrary{most}
\usepackage[version=4]{mhchem}
\usepackage{XyMTeX,chemfig}
\usepackage[bookmarksnumbered=true,bookmarksopen=true,hidelinks,breaklinks=true,pdfusetitle]{hyperref}
\usepackage[bigcode]{pxjahyper}
\renewcommand{\UrlBreaks}{\do\:\do\/\do\-\do\a\do\b\do\c\do\d\do\e\do\f\do\g\do\h\do\i\do\j\do\k\do\l\do\m\do\n\do\o\do\p\do\q\do\r\do\s\do\t\do\u\do\v\do\w\do\x\do\y\do\z\do\A\do\B\do\C\do\D\do\E\do\F\do\G\do\H\do\I\do\J\do\K\do\L\do\M\do\N\do\O\do\P\do\Q\do\R\do\S\do\T\do\U\do\V\do\W\do\X\do\Y\do\Z}
\usepackage{ltablex}
\usepackage{graphicx,xcolor}
\usepackage{tikz}
\usetikzlibrary{patterns,intersections,calc,arrows,decorations,angles,datavisualization,datavisualization.formats.functions,datavisualization.polar,automata}
\usepackage{listings,plistings}
\lstset{%
	language={C},%
	basicstyle={\normalsize\ttfamily},%
	identifierstyle={\normalsize},%
	commentstyle={\normalsize\ttfamily\bfseries\color[rgb]{0,0.4,0}},%
	keywordstyle={\normalsize\ttfamily\bfseries\color[rgb]{0,0,1}},%
	ndkeywordstyle={\normalsize},%
	stringstyle={\normalsize\ttfamily\bfseries\color[rgb]{0.4,0,0.4}},%
	frame={tb},%
	breaklines=true,%
	columns=[l]{fullflexible},%
	showstringspaces=false,%
	keepspaces=true,%
	numbers=left,%
	xrightmargin=0zw,%
	xleftmargin=0zw,%
	numberstyle={\scriptsize},%
	stepnumber=1,%
	numbersep=1zw,%
	lineskip=-0.4zw%
}
\usepackage[tone,extra]{tipa}
\usepackage{tipx}
\usepackage{siunitx}
\usepackage[at]{easylist}
\usepackage{amsthm}
\newtheoremstyle{mystyle}{\baselineskip}{\baselineskip}{\rmfamily}{}{\sffamily}{}{\newline}{}%Name, Space above, Space below, Body font, Indent amount, Theorem head font, Punctuation after theorem head, Space after theorem head: ' ' or \newline, Theorem head spec (can be left empty--meaning `normal')
\theoremstyle{mystyle}
\newtheorem{Axiom}{公理}[section]
\newtheorem{Definition}{定義}[section]
\newtheorem{Proposition}{命題}[section]
\newtheorem{Theorem}{定理}[section]
\newtheorem{Lemma}{補題}[section]
\newtheorem{Corollary}{系}[section]
\newtheorem{Conjecture}{予想}[section]
\renewcommand{\proofname}{\hspace{0.4zw}\textsf{証明}}
\renewcommand{\qedsymbol}{\(\blacksquare\)}
\makeatletter
	\renewenvironment{proof}[1][\proofname]{%
		\pushQED{\qed}%
		\normalfont%
		\trivlist%
		\item[#1]\ignorespaces\leavevmode\\
	}
	{%
		\popQED\endtrivlist\@endpefalse
	}%
	\def\delete#1from#2{% 
		\def\reserved@a{#1}% 
		\def\reserved@c{}% 
		\def\@elt##1{% 
			\def\reserved@b{##1}% 
			\ifx\reserved@a\reserved@b 
			\else 
				\@cons\reserved@c{{##1}}% 
			\fi
		}% 
		\csname cl@#2\endcsname 
		\expandafter\let\csname cl@#2\endcsname\reserved@c 
		\let\@elt\relax
	}%
\makeatother
\delete{footnote}from{chapter}
\usepackage{comment,csquotes}
\makeatletter
	\renewcommand{\theequation}{\thesection.\arabic{equation}}
	\@addtoreset{equation}{section}
	\renewcommand{\thetable}{\thesection.\arabic{table}}
	\@addtoreset{table}{section}
	\renewcommand{\thefigure}{\thesection.\arabic{figure}}
	\@addtoreset{figure}{section}
	\renewcommand{\thefootnote}{\arabic{footnote}}
	\AtBeginDocument{\renewcommand{\thelstlisting}{\thesection.\arabic{lstlisting}}}
	\@addtoreset{lstlisting}{section}
\makeatother
\newcolumntype{C}{>{\centering\arraybackslash}X}
\newcolumntype{L}{>{\raggedright\arraybackslash}X}
\newcolumntype{R}{>{\raggedleft\arraybackslash}X}
\allowdisplaybreaks[0]
%
\newcommand{\zwspace}{\hspace{1zw}\relax}
\renewcommand{\lstlistingname}{プログラム}
\newcommand{\captiondot}[1]{\caption{#1.}}
\newcommand{\figureinput}[4]{\begin{figure}[tbp]\centering\includegraphics[#1]{#2}\captiondot{#3}\label{fig:#4}\end{figure}}
\newcommand{\tableinput}[4]{\begin{table}[tbp]\centering\captiondot{#3}\label{tab:#4}\begin{tabular}{#1}#2\end{tabular}\end{table}}
\newcommand{\mathdisplaystyle}[1]{\(\displaystyle{#1}\)}
\newcommand{\Reference}[1]{\mathdisplaystyle{\ref{#1}}}
\newcommand{\Citation}[1]{\mathdisplaystyle{\cite{#1}}}
\newcommand{\Equationreference}[1]{\mathdisplaystyle{\parentheses{\ref{#1}}}}
\newcommand{\negativevalue}[1]{{-#1}}
\newcommand{\positivevalue}[1]{{+#1}}
\newcommand{\plusminusvalue}[1]{{\pm#1}}
\newcommand{\minusplusvalue}[1]{{\mp#1}}
\newcommand{\fraction}[2]{\displaystyle{\frac{\displaystyle{#1}}{\displaystyle{#2}}}}
\newcommand{\mathcomma}{\mathpunct{,\,}}
\newcommand{\mathsemicolon}{\mathpunct{;\,}}
\newcommand{\commas}{\mathpunct{\mathcomma\dotsc\mathcomma}}
\newcommand{\definitionequal}{\mathrel{:=}}
\newcommand{\definitionarrow}{\mathrel{\overset{\textrm{def}}{\Longleftrightarrow}}}
\newcommand{\ifarrow}{\mathrel{\Longleftarrow}}
\newcommand{\onlyifarrow}{\mathrel{\Longrightarrow}}
\newcommand{\ifandonlyifarrow}{\mathrel{\Longleftrightarrow}}
\newcommand{\shortfunction}[5]{\displaystyle{\mathord{#1}\colon#2\to#3\mathsemicolon#4\mapsto#5}}
\newcommand{\strutroot}[2]{\sqrt[#1]{\mathstrut{#2}}}
\newcommand{\alephnull}{\aleph_0}
\newcommand{\delivative}[1]{\displaystyle{\,d#1}}
\newcommand{\inversed}[1]{{#1}^{\negativevalue{1}}}
\newcommand{\parentheses}[1]{\left(#1\right)}
\newcommand{\braces}[1]{\left\{#1\right\}}
\newcommand{\squarebrackets}[1]{\left[#1\right]}
\newcommand{\mediumverticalline}[2]{#1\mathrel{}\middle|\mathrel{}#2}
\newcommand{\absolute}[1]{\left\lvert#1\right\rvert}
\newcommand{\norm}[2]{{\left\lVert#1\right\rVert}_{#2}}
\newcommand{\anglebrackets}[1]{\left\langle#1\right\rangle}
\newcommand{\floor}[1]{\left\lfloor#1\right\rfloor}
\newcommand{\ceil}[1]{\left\lceil#1\right\rceil}
\newcommand{\Diracbra}[1]{\left\langle#1\right\rvert}
\newcommand{\Diracket}[1]{\left\lvert#1\right\rangle}
\newcommand{\innerproduct}[2]{\anglebrackets{#1\mathcomma#2}}
\newcommand{\leftclosedrightopen}[2]{\left[{#1\mathcomma#2}\right)}
\newcommand{\leftopenrightclosed}[2]{\left({#1\mathcomma#2}\right]}
\newcommand{\gradient}[2]{\nabla_{#1}{#2}}
\newcommand{\divergence}[1]{\nabla\cdot#1}
\newcommand{\rotation}[1]{\nabla\times#1}
\DeclareMathOperator{\operatorgrad}{grad}
\DeclareMathOperator{\operatordiv}{div}
\DeclareMathOperator{\operatorrot}{rot}
\DeclareMathOperator{\realpart}{Re}
\DeclareMathOperator{\imaginarypart}{Im}
\DeclareMathOperator{\cardinality}{card}
\newcommand{\easymatrix}[1]{\mathord{\begin{pmatrix}#1\end{pmatrix}}}
\newcommand{\transpose}[1]{\fourIdx{t\!}{}{}{}{#1}}
\newcommand{\transposedmatrix}[1]{\transpose{\easymatrix{#1}}}
\newcommand{\combination}[2]{\fourIdx{}{#1}{}{#2}{\mathrm{C}}}
\newcommand{\permutation}[2]{\fourIdx{}{#1}{}{#2}{\mathrm{P}}}
\newcommand{\chemicalelement}[3]{\ce{_{#1}^{#2}#3}}
\title{\vspace{-25truemm}\Huge{SVMとCIELUV色空間を用いた\\1型2色覚,2型2色覚の色の\\見え方の分類}\vspace{55truemm}}
\author{\huge{坂井~法仁}}
\date{\LARGE{令和2年2月}\\ \vspace{30truemm}\LARGE{九州大学理学部物理学科\\ 情報理学コース}}
\begin{document}
	\urlstyle{rm}
	\interfootnotelinepenalty=10000000
	\maketitle
	\pagenumbering{roman}
	\tableofcontents
	%\listoftables
	%\listoffigures
	\clearpage
	\setcounter{page}{1}
	\pagenumbering{arabic}
	\chapter{序論}\label{chap:intro}
		\section{ヒトの眼球の構造}
			ヒトは眼球を使って物を見る.

			眼球の構造を図\Reference{fig:eyes}に示す.
			\figureinput{width=9.5truecm}{D:/a/figs/eye2.png}{眼球の構造\protect\cite[図-1]{Sotoyama1985}}{eyes}
			外界から入射する光は角膜(cornea)と前眼房(前房とも;anterior chamber)を通って水晶体(crystalline lens)に入り,そこで屈折する.
			更に硝子体(vitreous body)を通って網膜(retina)で結像する\footnote{水晶体のピント調整機能の低下で結像位置が網膜の前後に移動する現象が近視,遠視,老眼である.}.
			網膜にはL錐体(L cone),M錐体(M cone),S錐体(S cone),杆体(桿体とも;rod)の4種類の視細胞(photoreceptor)が分布している.
			3種類の錐体は中心窩(fovea)付近に局在していて,十分明るい環境で機能し,波長に対する各々の応答の程度(錐体分光感度;cone fundamental)の差異によって色を知覚できる.
			杆体は錐体と比べて光への感度が良く,暗所で機能する\cite{Hosoki2014}.
			杆体は1種類のみである為に色の知覚には無関係であると言われるが,錐体が十分に機能できない暗さで色覚に影響を及ぼしているとする研究もある\cite{Takeshita2014}.

			3種類の錐体の分光感度を波長毎にプロットした物が図\Reference{fig:fundament}である.
			横軸は光の波長,縦軸はその波長の光が目に入射した時に錐体がどれだけ応答するかを表していて,最大値が1になるように正規化されている.
			これは直接の実験結果ではなく,後述する等色実験(color matching experiment)の結果を線型変換して得られるデータである.
			国際照明委員会(Commission Internationale de l'Éclairage;CIE)が2015年に制定した,2\textdegree 視野のCIE2015\mathdisplaystyle{XYZ}色空間(CIE2015\mathdisplaystyle{XYZ} color space)
			\figureinput{width=\linewidth}{D:/a/figs/ConeFundamental.png}{CIE2015\mathdisplaystyle{XYZ}色空間(2\textdegree 視野)から導出される正規化錐体分光感度}{fundament}
			また,結果的にヒトが見る可視光のスペクトルは図\Reference{fig:spectrum}のようになる\footnote{Pythonライブラリ「Colour」を使用して描画した.}.
			\figureinput{width=10truecm}{D:/a/figs/spectrum.png}{CIE2012\mathdisplaystyle{XYZ}色空間(2\textdegree 視野)から導出される可視光のスペクトル}{spectrum}
		\section{色覚異常について}
			世の中には,先天的なX染色体異常がトリガーとなる色盲者\cite{Okabe2002a},色弱者\cite{Sunaga2017}や,加齢によって後天的に色の見え方が変化する人\cite{Kuriki2000}がいる.

			色盲者は1種類以上の錐体が先天的に欠損していて\footnote{或る錐体の個数は少ないが完全に欠損している訳ではないという人もいて,こちらは軽度であれば色弱に含めることがある.},色弱者は欠損は無いものの刺激の光の波長に対する錐体の感度分布が異なっている.
			これらを合わせると,国内では男性の約5\%,女性の約0.2\%が色盲者,色弱者である\cite[p.~4]{doctors2019}.
			L錐体が欠損している色盲を1型2色覚(protanopia),M錐体が欠損している色盲を2型2色覚(deuteranopia),S錐体が欠損している色盲を3型2色覚(tritanopia)という\footnote{色盲者を表す時は,protanopeのように接尾辞が-iaから-eに変化する.}.
			先天色覚異常の大多数は1型2色覚と2型2色覚である\cite[p.~9]{doctors2014}.

			加齢による色の見え方の変化は,核白内障(nuclear cataract)が原因の水晶体の黄変に伴い,光が錐体に届く前にスペクトルが歪められることによって起こる.
			特に短波長の光が黄変した水晶体に吸収されやすい為,青紫や青緑等の青系統の色の識別が難しくなる他,液体,炎や電光掲示板が見え辛くなる\cite{Ishihara1998}.
		\section{色覚シミュレーションについて}
			情報伝達の手段の一つとして色を用いることはよくある(図\Reference{fig:redtext1},図\Reference{fig:redtext2}等\footnote{図\Reference{fig:redtext1},図\Reference{fig:redtext2}のシミュレータはスマートフォンアプリケーション「色のシミュレータ」を使用した.}).
			マジョリティの3色覚者にとっては赤色が強調色や警告色として映るので,特に赤色はよく使用される.
			\figureinput{width=7.5truecm}{D:/a/figs/S__50315266.jpg}{左上:本学内の掲示①(撮影日:2020年1月17日),右上:元画像の1型2色覚の見え方のシミュレーション,左下:元画像の2型2色覚の見え方のシミュレーション,右下:元画像の3型2色覚の見え方のシミュレーション}{redtext1}
			\figureinput{width=7.5truecm}{D:/a/figs/S__50315270.jpg}{左上:本学内の掲示②(撮影日:2020年1月17日),右上:元画像の1型2色覚の見え方のシミュレーション,左下:元画像の2型2色覚の見え方のシミュレーション,右下:元画像の3型2色覚の見え方のシミュレーション}{redtext2}
			また,災害の危険が迫っている際に個々人の情報収集は迅速に行われるべきであり,色情報を加えることで一目で分かるようになっている(図\Reference{fig:yahoo}\footnote{図\protect\Reference{fig:yahoo}のシミュレータは「色のシミュレータ」を使用した.}\cite{Yahoo2020}).
			\begin{figure}[tbp]
				\centering
				\begin{tabular}{c}
					\begin{minipage}{0.5\linewidth}
						\centering
						\includegraphics[width=5.5truecm]{D:/a/figs/S__53149698.png}
					\end{minipage}
					\begin{minipage}{0.5\linewidth}
						\centering
						\includegraphics[width=5.5truecm]{D:/a/figs/S__53149700.jpg}
					\end{minipage} \\
					\begin{minipage}{0.06\linewidth}
						\vspace{10truemm}
					\end{minipage} \\
					\begin{minipage}{0.5\linewidth}
						\centering
						\includegraphics[width=5.5truecm]{D:/a/figs/S__53149701.jpg}
					\end{minipage}
					\begin{minipage}{0.5\linewidth}
						\centering
						\includegraphics[width=5.5truecm]{D:/a/figs/S__53149702.jpg}
					\end{minipage}
				\end{tabular}
				\captiondot{左上:災害情報の危険度別の色,右上:元画像の1型2色覚の見え方のシミュレーション,左下:元画像の2型2色覚の見え方のシミュレーション,右下:元画像の3型2色覚の見え方のシミュレーション}\label{fig:yahoo}
			\end{figure}

			しかし,1型2色覚,2型2色覚者にとって赤色は強調色になり得ないし,災害時に慌てていれば似た色を見誤って被害に遭う可能性もある.
			故に,特に情報伝達としての色の組み合わせは色覚異常があることによって識別しづらくならない方が良い.
			また,色覚異常の知識がないデザイナーが例えば1型2色覚のことを知って「彼らは赤色が見えない」と思い込むと,そのデザイナーが単独でデザインした時に確かに赤色の使用は減るかもしれないが,実際にそれで色覚異常当事者たちにとって識別しやすくなっているかどうか分からない.
			以上のことを踏まえると,情報伝達として色を用いる際は複数の当事者に色の見え方を聞きながらデザインすることになる.
			ただ,この調整がデザイナー,当事者双方に大きな負担になり得るのは想像に難くない.

			この負担を減らせるツールとして色盲のシミュレータ(色覚シミュレータとも呼ばれる)がある.
			色覚シミュレータは多数あるが,その多くはBrattelらが開発したアルゴリズム\cite{Brettel1997}をベースにしている.
			Brattelらのアルゴリズムは,画像の\mathdisplaystyle{RGB}値を\mathdisplaystyle{LMS}色空間\footnote{3種類の錐体の応答値を3次元空間にプロットすることで構成される.}に座標変換してから,各色覚特性に合わせた適切な処理を行い,\mathdisplaystyle{RGB}色空間に逆変換する.
			シミュレータを用いることで,ディスプレイに映し出された画像をリアルタイムに色変換させることが可能になり,マジョリティである3色覚者でも2色覚の見え方を擬似体験できるようになった.
			\figureinput{width=10truecm}{D:/a/figs/S__53141507.jpg}{上:可視光のスペクトル画像,中:元画像の1型2色覚の見え方のシミュレーション,下:元画像の2型2色覚の見え方のシミュレーション}{spect}
			図\Reference{fig:spect}は可視光のスペクトル画像\footnote{図\protect\Reference{fig:spectrum}と同じ物.}とそれに1型2色覚,2型2色覚のシミュレーションを実行した結果である\footnote{図\protect\Reference{fig:spect}のシミュレータは「色のシミュレータ」を使用した.}.
			
			1型2色覚と2型2色覚の色の見え方は似ている.
			この原因として,図\Reference{fig:fundament}から分かるように,L錐体とM錐体の錐体分光感度の分布は近くにあり,\mathdisplaystyle{LMS}色空間は\mathdisplaystyle{L}軸の基底ベクトルと\mathdisplaystyle{M}軸の基底ベクトルが近い向きを向いている斜交座標系になっている(図\Reference{fig:LMS}参照).
			\begin{figure}[tbp]
				\centering
				\begin{tikzpicture}
					\coordinate (O) at (0,0);
					\coordinate (A) at (-0.5,-1);
					\coordinate (B) at (0.5,-1);
					\coordinate (C) at (1.7320508,1);
					\draw[->, red, very thick] (O)--(A);
					\draw[->, green, very thick] (O)--(B);
					\draw[->, blue, very thick] (O)--(C);
					\draw (O)node[left]{\mathdisplaystyle{\mathrm{O}}};
					\draw (A)node[below left]{\mathdisplaystyle{L}};
					\draw (B)node[below right]{\mathdisplaystyle{M}};
					\draw (C)node[above right]{\mathdisplaystyle{S}};
				\end{tikzpicture}
				\captiondot{\protect\mathdisplaystyle{LMS}色空間の座標軸のイメージ(実際の座標軸同士の為す角はこの通りでない)}\label{fig:LMS}
			\end{figure}
			1型2色覚と2型2色覚の色の見え方の類似性はここからきていると考えられる.

			なお,色覚シミュレータが実際の色盲者の見え方を忠実に再現できているかを検証するのは難しい.
			色盲の理論はあるので3色覚者はそれから2色覚者の見え方を予測することはできるが実体験しようがなく,2色覚者はシミュレーションで色が自分の普段の見え方に変化したかを感知できない為である.
			従って,3色覚者はシミュレータを過信せず,当事者から適宜意見を貰うことになる.
		\section{カラーユニバーサルデザインについて}
			Brattelらのアルゴリズムの恩恵は他にもある.
			それは,色盲者や色弱者でも識別しやすい色の組み合わせを用いて情報伝達を行うカラーユニバーサルデザインを考慮しやすくなったということである.
			近年,多くの色覚異常当事者の協力の下でカラーユニバーサルデザイン配色が考案される(\cite[pp.~1091--1099]{Okabe2002c},\cite{cudo2018}等),どんな人でも適切に識別できる組み合わせのチョークが開発される\cite{Rikagaku2020},JIS Z 9013(図記号―安全色及び安全標識―安全色の色度座標の範囲及び測定方法)が改正される\cite{Nakano2018}等の活動が行われてきた.
			今後もこれらの活動が広がっていく中で,色覚シミュレータを上手く活用すれば当事者と意見を摺り合わせる回数を減らせる可能性が高くなり,デザイナー,当事者双方の負担を軽減でき得るからである.
		\section{本論文の目的}
			しかし,Brattelらのアルゴリズムを実装すると,前処理として可視光波長域上での数値積分を3回,3次正方行列の逆行列の計算を1回,3次元ベクトルの外積を2回実行し,更に各色毎に3次正方行列と3次元ベクトルの乗算を3回ずつ行わなければならず,24 bitの\mathdisplaystyle{RGB}色空間で\mathdisplaystyle{2^{24}}色\footnote{\mathdisplaystyle{2^{24}=16{,}777{,}216}.}全てに亘って変換を掛けながら当事者と見え方を摺り合わせていくのは大変な時間を要する.
			その上,実装や改良に際して色彩学や生理学に関する必要な知識が多く,専門外の人が取り組みにくくなっていると言わざるを得ない.

			そこで本論文では,人口が比較的多く,色の見え方が類似している1型2色覚,2型2色覚に着目する.
			この2種類の色覚で識別しやすい配色を予め求めていれば実際のデザイン時に繰り返し活用できるので,機械学習を用いて効率的に配色セット設計を行う.
			本学芸術工学研究院デザイン人間科学部門の須長正治准教授が研究されているnatural color system(NCS)に関する色のデータセット\footnote{本来は1950色であるが,手元に届いた時点でS2075-Y60RとS3050-R20Bが同梱されていなかったという.従って,以下で「NCS」と言う時はこれらを除いた1948色を指す.}を用いて,NCSのAdobe\mathdisplaystyle{RGB}\footnote{1998年考案.}座標,サポートベクトルマシン(support vector machine;SVM),Gauß過程(Gaussian process;GP)によるBayes最適化(Bayesian optimization)でNCSの色を1型2色覚,2型2色覚それぞれの色の見え方に分類できることを示す.
			更に,それで得たパラメータと1948色のNCS全体を用いて,Adobe\mathdisplaystyle{RGB}色空間上の非負格子点(\mathdisplaystyle{2^{24}}色)全てにおける各2色覚での色の見え方を予測する.
			各非負格子点の正解ラベルが未知であることを踏まえ,色差(color difference)\footnote{2つの色の区別しやすさに関する定量的指標.}が定義されている均等色空間(uniform color space)の一つ,CIE1976\mathdisplaystyle{L^\ast u^\ast v^\ast}色空間(CIELUV色空間)\cite[p.~64]{Yaguchi2017b}に座標変換して予測ラベルの代表色との色差による精度評価も行う.
	\chapter{色空間の歴史と概要}
		既に第\Reference{chap:intro}章でCIE2015\mathdisplaystyle{XYZ}色空間(2\textdegree 視野),\mathdisplaystyle{LMS}色空間,Adobe\mathdisplaystyle{RGB}色空間,CIELUV色空間が登場しているが,本章で一度これらを整理する.
		なお,2015年の複数のデータや色空間間座標変換の式は入手できなかった為,以下ではそれよりもやや古い物を掲載する.
		\section{色を数値化する}
			電磁波は波長を持っているのみで色が付いている訳ではない.
			色を知覚しているのは動物側の都合である.
			また,同じ光であってもそれを見る動物,人,環境によって知覚される色は大きく変わる.
			色を科学的に取り扱う為には色を数値化する必要が生じた.
		\section{分光視感効率}
			ヒトが感じる色の明るさ(分光視感効率,比視感度;luminous efficiency)はその波長によって異なる.
			例えば,同じエネルギーでも緑色は明るく見え,紫外域,赤外域には明るさを感じない.
			分光視感効率は波長の非負有界連続関数\mathdisplaystyle{V\colon\parentheses{0,\infty}\to\squarebrackets{0,1};\lambda\mapsto V\parentheses{\lambda}}で表される.
			ここで,\mathdisplaystyle{\lambda}\,[nm]は光の波長である.
			CIE1924分光視感効率\mathdisplaystyle{V\parentheses{\lambda}}(暗い青色の実線)とJudd--Vos修正分光視感効率\mathdisplaystyle{V_\textrm{M}\parentheses{\lambda}}\cite{Vos1978}(暗い黄色の破線)を図\Reference{fig:le}に示す.
			\figureinput{width=\linewidth}{D:/a/figs/le.png}{CIE1924分光視感効率\mathdisplaystyle{V\parentheses{\lambda}}(暗い青色の実線)とJudd--Vos修正分光視感効率\mathdisplaystyle{V_\textrm{M}\parentheses{\lambda}}(暗い黄色の破線)}{le}
			\mathdisplaystyle{V\parentheses{\lambda}},\mathdisplaystyle{V_\textrm{M}\parentheses{\lambda}}は共に\mathdisplaystyle{\lambda\sim{}}\SI{555}{nm}で極大かつ最大となる.
			修正によって\mathdisplaystyle{\lambda\sim{}}\SI{425}{nm}での感度がより高く見積もられることになった.

			光の放射が波長\mathdisplaystyle{\lambda}について放射束(radiant flux)\mathdisplaystyle{\greekPHI_\textrm{r}\parentheses{\lambda}}\,[W]を持つならば,測光量(luminous quantity)\mathdisplaystyle{\greekPHI_\textrm{l}}\,[lm]は
			\begin{equation*}
				\greekPHI_\textrm{l}=683\int_0^\infty{\greekPHI_\textrm{r}\parentheses{\lambda}V\parentheses{\lambda}}\,d\lambda
			\end{equation*}
			で求まる\footnote{\mathdisplaystyle{[\textrm{lm}]=[\textrm{cd}\,\textrm{sr}]}である.}.
			比例定数の683\,lm\,\mathdisplaystyle{\textrm{W}^{\negativevalue{1}}}は国際単位系(international system of units)で定められた発光効率(luminous efficiency)\mathdisplaystyle{K_\textrm{cd}}である.
		\section{等色関数と\mathdisplaystyle{RGB}色空間}
			マジョリティは3種類の錐体を持つので,波長の異なる3種類の単色光を上手く混ぜ合わせれば任意の光を再現できると言えそうである.
			適当なスペクトルを持つ色を\mathdisplaystyle{\Diracket{C}}とし,この輝度(luminance)を\mathdisplaystyle{L_C}\,[cd\,\mathdisplaystyle{\textrm{m}^\negativevalue{2}}]とする.
			原刺激(reference stimulus)\mathdisplaystyle{\Diracket{R}},\mathdisplaystyle{\Diracket{G}},\mathdisplaystyle{\Diracket{B}}によって\mathdisplaystyle{\Diracket{C}}と同じに見える色が再現できた時,条件等色(metamerism)といい,
			\begin{equation*}
				L_C\Diracket{C}\equiv \easymatrix{L_R & L_G & L_B}\easymatrix{\Diracket{R} \\ \Diracket{G} \\ \Diracket{B}}
			\end{equation*}
			と表す.
			ここで得られる\mathdisplaystyle{\parentheses{L_R,L_G,L_B}}にある定数倍を施して得られる\mathdisplaystyle{\parentheses{R,G,B}}の空間を\mathdisplaystyle{RGB}色空間という.
			また,このようにして,調べたい色\mathdisplaystyle{\Diracket{C}}のスペクトルを変えながら条件等色となる\mathdisplaystyle{L_R},\mathdisplaystyle{L_G},\mathdisplaystyle{L_B}を調べることを等色実験という.
			更に,\mathdisplaystyle{\Diracket{C}=\Diracket{\lambda\,\textrm{[nm]}}}の時(則ち調べたい色が単色光の時)のこの\mathdisplaystyle{R},\mathdisplaystyle{G},\mathdisplaystyle{B}を等色関数(color maching function)といい,\mathdisplaystyle{\bar{r}\parentheses{\lambda}},\mathdisplaystyle{\bar{g}\parentheses{\lambda}},\mathdisplaystyle{\bar{b}\parentheses{\lambda}}と表す.
			CIE1931での\mathdisplaystyle{\Diracket{R}},\mathdisplaystyle{\Diracket{G}},\mathdisplaystyle{\Diracket{B}}はそれぞれ\SI{700.0}{nm},\SI{546.1}{nm},\SI{435.8}{nm}で定められている.
			Stiles--Burchの等色関数\mathdisplaystyle{\bar{r}\parentheses{\lambda}},\mathdisplaystyle{\bar{g}\parentheses{\lambda}},\mathdisplaystyle{\bar{b}\parentheses{\lambda}}(\cite{Stiles1955,Stiles1959}より)を図\Reference{fig:matching}に示す\footnote{Stiles--Burchの等色関数における\mathdisplaystyle{\Diracket{R}},\mathdisplaystyle{\Diracket{G}},\mathdisplaystyle{\Diracket{B}}はそれぞれ\SI{648.6}{nm},\SI{526.4}{nm},\SI{445.3}{nm}であり,CIE1931\mathdisplaystyle{XYZ}のそれとは異なる.}.
			\figureinput{width=\linewidth}{D:/a/figs/RGBcmf.png}{Stiles--Burchの等色関数(暗い色の線が2\textdegree 視野,明るい色の線が10\textdegree 視野)}{matching}
			暗い色の線が2\textdegree 視野,明るい色の線が10\textdegree 視野である\footnote{2\textdegree 視野は中心窩付近に対応する視野であるが,黄斑(macula)が短波長の光を吸収しやすいことから,その影響を避ける為に10\textdegree 視野を用いることもある.}.
			\mathdisplaystyle{\lambda\sim{}}\SI{500}{nm}で\mathdisplaystyle{\bar{r}\parentheses{\lambda}<0}となっているのは特筆すべき点である\footnote{\mathdisplaystyle{\bar{g}\parentheses{\lambda}},\mathdisplaystyle{\bar{b}\parentheses{\lambda}}についてもこれらが負の値を取るような\mathdisplaystyle{\lambda}は存在する.}.
			これは,例えば2\textdegree 視野で
			\begin{equation*}
				L_{500\,\textrm{nm}}\Diracket{500\,\textrm{nm}}\equiv \easymatrix{\negativevalue{0.29500} & 0.49060 & 0.10749}\easymatrix{\Diracket{648.6\,\textrm{nm}} \\ \Diracket{526.4\,\textrm{nm}} \\ \Diracket{445.3\,\textrm{nm}}}
			\end{equation*}
			ということになっているのだが,実験の上では
			\begin{equation*}
				L_{500\,\textrm{nm}}\Diracket{500\,\textrm{nm}}+0.29500\Diracket{648.6\,\textrm{nm}}\equiv0.49060\Diracket{526.4\,\textrm{nm}}+0.10749\Diracket{445.3\,\textrm{nm}}
			\end{equation*}
			で条件等色になったことを意味している.

			測光量と同様に,
			\begin{equation*}
				C\Diracket{C}=\int_0^\infty{a\parentheses{\lambda}\Diracket{\lambda\,\textrm{[nm]}}}\,d\lambda
			\end{equation*}
			を持つならば,その光の色は
			\begin{equation*}
				\easymatrix{R \\ G \\ B}=\int_0^\infty{a\parentheses{\lambda}\easymatrix{\bar{r}\parentheses{\lambda} \\ \bar{g}\parentheses{\lambda} \\ \bar{b}\parentheses{\lambda}}}\,d\lambda
			\end{equation*}
			で表せる.
		\section{CIE1931\mathdisplaystyle{XYZ}色空間}
			\mathdisplaystyle{RGB}色空間は等色関数の性質から負の値も考える必要がある.
			しかし,1930年前後は色の計算を手計算で行っていて,負の値が多くの計算間違いを引き起こした為,全てが非負となる別の色空間が求められるようになっていった.
			そこで定義されたのがCIE1931\mathdisplaystyle{XYZ}色空間である\footnote{CIE2012\mathdisplaystyle{XYZ},CIE2015\mathdisplaystyle{XYZ}はCIE1931\mathdisplaystyle{XYZ}をより新しくした物である.}.
			CIE1931\mathdisplaystyle{XYZ}は次の3つの要請を全て満足するように構築された.
			\begin{align*}
				&\textrm{①}\exists!\,{M_{RGB\to XYZ}\in\blackboardR^{3\times3}}\mathcomma\forall\lambda\in\parentheses{0,\infty}\mathcomma\easymatrix{X\parentheses{\lambda} \\ Y\parentheses{\lambda} \\ Z\parentheses{\lambda}}=M_{RGB\to XYZ}\easymatrix{\bar{r}\parentheses{\lambda} \\ \bar{g}\parentheses{\lambda} \\ \bar{b}\parentheses{\lambda}}\textrm{,}\\
				&\textrm{②}\forall\lambda\in\parentheses{0,\infty}\mathcomma\easymatrix{X\parentheses{\lambda} \\ Y\parentheses{\lambda} \\ Z\parentheses{\lambda}}\geq\bm{0}\textrm{,}\\
				&\textrm{③}\forall\lambda\in\parentheses{0,\infty}\mathcomma Y\parentheses{\lambda}=V\parentheses{\lambda}\quad\parentheses{V\parentheses{\lambda}\textrm{:CIE1924分光視感効率}}\textrm{.}
			\end{align*}
			①は変換のしやすさの為,②は先述の通り,③は\mathdisplaystyle{Y\parentheses{\lambda}}に意味を与えたかった為である.
			結果的に変換行列\mathdisplaystyle{M}は
			\begin{equation*}
				M_{RGB\to XYZ}=\fraction{1}{0.17697}\easymatrix{0.49000 & 0.31000 & 0.20000 \\ 0.17697 & 0.81240 & 0.01063 \\ 0.00000 & 0.01000 & 0.99000}
			\end{equation*}
			で定められた.
			CIE1931\mathdisplaystyle{XYZ}(2\textdegree 視野,暗い実線)とCIE2015\mathdisplaystyle{XYZ}(2\textdegree 視野,明るい破線)を図\Reference{fig:XYZ}に示す.
			\figureinput{width=\linewidth}{D:/a/figs/XYZcmf.png}{CIE1931\mathdisplaystyle{XYZ}(2\textdegree 視野,暗い実線)とCIE2015\mathdisplaystyle{XYZ}(2\textdegree 視野,明るい破線)}{XYZ}
			\mathdisplaystyle{X\parentheses{\lambda}}の極大点が2点存在する(\mathdisplaystyle{\lambda\sim{}}\SI{440}{nm}と\mathdisplaystyle{\lambda\sim{}}\SI{600}{nm})ことが特徴である.
			修正によって全体的に感度が向上した.
		\section{\mathdisplaystyle{LMS}色空間}
			1930年頃は錐体の感度を直接測定する方法が無く,錐体分光感度をCIE1931\mathdisplaystyle{XYZ}からの線型変換で推定した.
			則ち,
			\begin{equation*}
				\exists!\,{M_{XYZ\to LMS}\in\blackboardR^{3\times3}}\mathcomma\forall\lambda\in\parentheses{0,\infty}\mathcomma\easymatrix{L\parentheses{\lambda} \\ M\parentheses{\lambda} \\ S\parentheses{\lambda}}=M_{XYZ\to LMS}\easymatrix{X\parentheses{\lambda} \\ Y\parentheses{\lambda} \\ Z\parentheses{\lambda}}
			\end{equation*}
			と仮定し,
			\begin{equation*}
				M_{RGB\to XYZ}\sim\easymatrix{0.21058 & 0.85509 & \negativevalue{0.03970} \\ \negativevalue{0.41707} & 1.17726 & 0.07863 \\ 0.00000 & 0.00000 & 0.51684}
			\end{equation*}
			を得た.
			初めに掲載した図\Reference{fig:fundament}はCIE2015\mathdisplaystyle{XYZ}に対してこの線型変換を施した結果だったのである.
		\section{\mathdisplaystyle{xy}色度図}
			今まで述べてきたように色は3次元空間上で表現される.
			しかし,実際にデザインする時に3次元だとイメージが難しいという課題がある.
			そこで,\begin{align*}
				&x=\fraction{X}{X+Y+Z}\textrm{,}\\
				&y=\fraction{Y}{X+Y+Z}
			\end{align*}
			で2次元に落とす\mathdisplaystyle{xy}色度図(\mathdisplaystyle{xy} color diagram)が開発された(図\Reference{fig:xydiagram})\footnote{図\Reference{fig:xydiagram}は「Colour」を使用して描画した.}.
			\figureinput{width=\linewidth}{D:/a/figs/xy.png}{CIE1931\mathdisplaystyle{xy}色度図}{xydiagram}

			\mathdisplaystyle{xy}色度図上の点\mathdisplaystyle{\parentheses{x,y}=\parentheses{1/3,1/3}}は無彩色であり,白色点(white point)と呼ばれ,そこから離れる程彩度が向上する.
			外周は純色であり\footnote{如何なるディスプレイ,プリンタを用いても純色の再現は不可能であるので,想像する他に無い.},上に凸な曲線部分(スペクトル軌跡;spectrum locus)は光のスペクトルに一致する.
			下の線分を純紫軌跡(line of purples)といい,スペクトルには存在しない.
			\mathdisplaystyle{xy}色度図上の任意の2色\mathdisplaystyle{\Diracket{c_\alpha}},\mathdisplaystyle{\Diracket{c_\beta}}の混色はその間の線分上にあり
			\begin{equation*}
				\Diracket{c\parentheses{t}}^{xy}=\parentheses{1-t}\Diracket{c_\alpha}^{xy}+t\Diracket{c_\beta}^{xy}\quad\parentheses{t\in\parentheses{0,1}}
			\end{equation*}
			と表せ,任意の色\mathdisplaystyle{\Diracket{c}}の補色は白色点を中点とする線分の反対側
			\begin{equation*}
				\Diracket{c'}^{xy}=\negativevalue{\Diracket{c}^{xy}}+\fraction{2}{3}\easymatrix{1 \\ 1}
			\end{equation*}
			に位置する.
			ヒトが見る色は全てこの領域で網羅されていることから,色を定量的に取り扱うことが容易になり,デザインにも大きく貢献することになった.
			\subsection{混同色理論}
				\mathdisplaystyle{xy}色度図を用いると,色盲者が識別できない混同色の理論の記述が可能になる.
				1型2色覚,2型2色覚,3型2色覚にはそれぞれ混同色中心(copunctual point)と呼ばれる1点を\mathdisplaystyle{xy}色度図上に持ち,そこから放射状に混同色線(confusion line)が伸びている.
				混同色中心の色度座標は表\Reference{tab:copunct}に示す値である\cite[Table 1]{Fomins2011}.
				\tableinput{l|rr}{ & \(x\) & \(y\) \\ \hline
					1型2色覚 & 0.7465 & 0.2535 \\
					2型2色覚 & 1.4000 & \mathdisplaystyle{\negativevalue{0.4000}} \\
					3型2色覚 & 0.1748 & 0.0000}{混同色中心の色度座標}{copunct}
				実際に混同色線を描いてシミュレーションすると図\Reference{fig:xysimulate}が得られ,主張は概ね正しいことが分かる\footnote{図\Reference{fig:xysimulate}のシミュレータは「色のシミュレータ」を使用した.}.
				\begin{figure}[tbp]
					\centering
					\begin{tabular}{c}
						\begin{minipage}{0.5\linewidth}
							\centering
							\includegraphics[width=5.5truecm]{D:/a/figs/IMG_0137.JPG}
						\end{minipage}
						\begin{minipage}{0.5\linewidth}
							\centering
							\includegraphics[width=5.5truecm]{D:/a/figs/IMG_0138.JPG}
						\end{minipage} \\
						\begin{minipage}{0.06\linewidth}
							\vspace{10truemm}
						\end{minipage} \\
						\begin{minipage}{0.5\linewidth}
							\centering
							\includegraphics[width=5.5truecm]{D:/a/figs/IMG_0140.JPG}
						\end{minipage}
						\begin{minipage}{0.5\linewidth}
							\vspace{10truemm}
						\end{minipage}
					\end{tabular}
					\captiondot{左上:1型2色覚の混同色線とシミュレーション,右上:2型2色覚の混同色線とシミュレーション,下:3型2色覚の混同色線とシミュレーション}\label{fig:xysimulate}
				\end{figure}
		\section{s\mathdisplaystyle{RGB}とAdobe\mathdisplaystyle{RGB}}
			s\mathdisplaystyle{RGB}とAdobe\mathdisplaystyle{RGB}の出発点はCIE1931\mathdisplaystyle{XYZ}である.
		\section{色差を数値化する}
			色の数値化は\mathdisplaystyle{XYZ}色空間を筆頭に成功を収めた.
			そこで次に検討されたこととして,任意の2色の間の色差を定量的に取り扱う方法が挙げられる.
			CIE1931\mathdisplaystyle{XYZ}や\mathdisplaystyle{xy}色度図で評価されたが,これらでは知覚の均一性が弱く,色差計算には不向きであることがMacAdamによって明らかになった\cite{MacAdam1942}.
			図\Reference{fig:MacAdam}は\mathdisplaystyle{xy}色度図にMacAdamの楕円を描き加えた物である\footnote{図\Reference{fig:MacAdam}は「Colour」を使用して描画した.}.
			\figureinput{width=\linewidth}{D:/a/figs/MacAdam.png}{\mathdisplaystyle{xy}色度図とMacAdamの楕円}{MacAdam}
			MacAdamの楕円は,ある色から少しずらしても色の変化を知覚できない領域を楕円で表し,相似比10で拡大した物で,もし色空間が完全に均等色空間であるならば,どの楕円も必ず真円になる.
			\mathdisplaystyle{xy}色度図では特に\mathdisplaystyle{y}座標が大きい所で縦に引き伸ばされていることが分かる.
			そこで,知覚の均一性を考慮した均等色空間が複数考案され,現在でも改良が続けられている.
		\section{CIE1976\mathdisplaystyle{L^\ast a^\ast b^\ast}色空間}
			現代でも比較的よく使われる均等色空間の一つがCIE1976\mathdisplaystyle{L^\ast a^\ast b^\ast}色空間(CIELAB色空間)である.
			CIELABの導出には,\mathdisplaystyle{XYZ}のデータとは別に白色光源(white light source)を設定する必要がある.
			CIEが既定する白色光源は複数あるが,タングステンランプを模したCIE標準光源A(CIE standard illuminant A)と晴天の昼間の太陽を模したCIE標準光源D65(CIE standard illuminant D65)が用いられることが多い.
			波長毎の2つの標準光源の相対分光分布を図\Reference{fig:illum}に示す.
			\figureinput{width=\linewidth}{D:/a/figs/illuminant.png}{標準光源の相対分光分布(水色の実線がD65,オレンジ色の破線がA)}{illum}
			2つの色度座標は表\Reference{tab:illum}で定められている.
			\tableinput{l|rr}{ & \(x_\textrm{n}\) & \(y_\textrm{n}\) \\ \hline
				A & 0.4475 & 0.4074 \\
				D65 & 0.3128 & 0.3291}{標準光源の色度座標}{illum}

			CIELAB空間の座標\mathdisplaystyle{\parentheses{L^\ast,a^\ast,b^\ast}}は以下のようにして求められる.
			まず,白色光源の\mathdisplaystyle{Y_\textrm{n}}が100になるように調整する(AとD65では表\Reference{tab:illum2}のようになる).
			\tableinput{l|rrr}{ & \(X_\textrm{n}\) & \(Y_\textrm{n}\) & \(Z_\textrm{n}\) \\ \hline
				A & 109.8470 & 100.0000 & 35.6273 \\
				D65 & 95.0461 & 100.0000 & 108.7865}{標準光源の\mathdisplaystyle{XYZ}}{illum2}
			\mathdisplaystyle{Y}の最大値が100になるように注意しながら求めたい色の\mathdisplaystyle{XYZ}も調整する.
			次に関数\mathdisplaystyle{f}を
			\begin{equation*}
				f\parentheses{\eta}=\begin{cases}
					\fraction{1}{3}\parentheses{\fraction{29}{6}}^2\eta+\fraction{4}{29} & \parentheses{\eta\leq\parentheses{\fraction{6}{29}}^3} \\
					\sqrt[3]{\eta} & \parentheses{\textit{otherwise}}
				\end{cases}
			\end{equation*}
			で定義する.
			この時,
			\begin{align*}
				&L^\ast\parentheses{X,Y,Z}=116f\parentheses{\fraction{Y}{Y_\textrm{n}}}-16\textrm{,}\\
				&a^\ast\parentheses{X,Y,Z}=500\braces{f\parentheses{\fraction{X}{X_\textrm{n}}}-f\parentheses{\fraction{Y}{Y_\textrm{n}}}}\textrm{,}\\
				&b^\ast\parentheses{X,Y,Z}=200\braces{f\parentheses{\fraction{Y}{Y_\textrm{n}}}-f\parentheses{\fraction{Z}{Z_\textrm{n}}}}
			\end{align*}
			となる.
			任意の2色\mathdisplaystyle{\Diracket{c_\alpha}},\mathdisplaystyle{\Diracket{c_\beta}}間の色差\mathdisplaystyle{\greekDELTA E^{L^\ast a^\ast b^\ast}\parentheses{\Diracket{c_\alpha},\Diracket{c_\beta}}}はCIELAB上のEuclid距離で
			\begin{equation}
				\greekDELTA E^{L^\ast a^\ast b^\ast}\parentheses{\Diracket{c_\alpha},\Diracket{c_\beta}}=\norm{\Diracket{c_\beta}^{L^\ast a^\ast b^\ast}-\Diracket{c_\alpha}^{L^\ast a^\ast b^\ast}}{2}\label{eq:LAB}
			\end{equation}
			で表される.

			CIELABは上半球のようになっている(図\Reference{fig:LAB}参照).
			\begin{figure}[tbp]
				\centering
				\begin{tabular}{c}
					\begin{minipage}{0.5\linewidth}
						\centering
						\includegraphics[width=7.5truecm]{D:/a/figs/CIELAB_color_space_front_view.png}
					\end{minipage}
					\begin{minipage}{0.5\linewidth}
						\centering
						\includegraphics[width=7.5truecm]{D:/a/figs/CIELAB_color_space_top_view.png}
					\end{minipage}
				\end{tabular}
				\captiondot{左:CIELABを\mathdisplaystyle{b^\ast}軸負の方向から正の方向を見た様子,右:CIELABを\mathdisplaystyle{L^\ast}軸正の方向から負の方向を見た様子.Wikipediaより引用}\label{fig:LAB}
			\end{figure}
			\mathdisplaystyle{L^\ast}が増えると明るくなり,\mathdisplaystyle{a^\ast}が増えると赤みが増し,\mathdisplaystyle{b^\ast}が増えると黄みが増す.
			CIELABの色域はヒトが知覚できる色域よりも広く,仮想的な色を多く持っている.

			CIELABはそれ以降の知覚の均一性改善の為に開発される様々な均等色空間の基礎になっている.
		\section{CIELUV色空間}
			CIELABと同時に開発されたのがCIELUVである.
			均等色空間CIE1960UCSの改良版として考案された.

			導出法は\mathdisplaystyle{XYZ}の調整まではCIELABと全く同じである.
			そこから
			\begin{equation*}
				L^\ast\parentheses{X,Y,Z}=\begin{cases}
					\parentheses{\fraction{29}{3}}^3\fraction{Y}{Y_\textrm{n}} & \parentheses{\fraction{Y}{Y_\textrm{n}}\leq\parentheses{\fraction{6}{29}}^3} \\
					116\parentheses{\fraction{Y}{Y_\textrm{n}}}^3-16 & \parentheses{\textit{otherwise}}
				\end{cases}
			\end{equation*}
			で\mathdisplaystyle{L^\ast}を求める.
			次に\mathdisplaystyle{u'},\mathdisplaystyle{v'}を
			\begin{align*}
				&u'\parentheses{X,Y,Z}=\fraction{4X}{X+15Y+3Z}\\
				&v'\parentheses{X,Y,Z}=\fraction{9Y}{X+15Y+3Z}
			\end{align*}
			で定め,\mathdisplaystyle{u^\ast},\mathdisplaystyle{v^\ast}を
			\begin{align*}
				&u^\ast\parentheses{X,Y,Z}=13L^\ast\parentheses{u'-u'_\textrm{n}}\\
				&v^\ast\parentheses{X,Y,Z}=13L^\ast\parentheses{v'-v'_\textrm{n}}
			\end{align*}
			とする.
			色差\mathdisplaystyle{\greekDELTA E^{L^\ast u^\ast v^\ast}\parentheses{\Diracket{c_\alpha},\Diracket{c_\beta}}}は式\Equationreference{eq:LAB}と同様に
			\begin{equation}
				\greekDELTA E^{L^\ast u^\ast v^\ast}\parentheses{\Diracket{c_\alpha},\Diracket{c_\beta}}=\norm{\Diracket{c_\beta}^{L^\ast u^\ast v^\ast}-\Diracket{c_\alpha}^{L^\ast u^\ast v^\ast}}{2}\label{eq:LUV}
			\end{equation}
			で表される.

			\mathdisplaystyle{u'v'}平面とMacAdamの楕円を図\Reference{fig:luvMacAdam}に示す\footnote{図\Reference{fig:luvMacAdam}は「Colour」を使用して描画した.}.
			\figureinput{width=\linewidth}{D:/a/figs/LuvMacAdam.png}{\mathdisplaystyle{u'v'}平面とMacAdamの楕円}{luvMacAdam}
			図\Reference{fig:MacAdam}と比べて知覚の均一性が改善していることが分かる.
			本論文で使用する均等色空間はCIELUVである.
		\section{物体色ベースの色空間}
			ここまで光源色ベースの色空間を議論してきた.
			一方で,物体色ベースの色空間を考えることも多い.
			発光しない物体は外部の光を一部反射することでその物体に色があるように見える.
			従って,物体色を光源色と同様に取り扱う為には白色光源を置いて測色する必要がある.
		\section{Munsell色空間}
			物体色ベースの色空間でメジャーな物としてMunsell色空間がある\footnote{一般的にはMunsell表色系(Munsell color system)という.}.
			Munsell色空間は色相(hue),明度(lightness,Munsell色空間ではvalueという),彩度(saturation,Munsell色空間ではchromaという)の3つで表す.
			彩度を動径座標,色相を角度座標とする円柱座標系と見做すとMunsell色空間は図\Reference{fig:Munsell}のようになる.
			\figureinput{width=5truecm}{D:/a/figs/munsell3.png}{Munsell色空間のイメージ\protect\cite{Fukue2010}}{Munsell}
			彩度が最大となる明度が色相毎に異なっていて,横から見ると歪な形状になっている.
			また,Munsell色空間はグラデーションにならず,任意の領域はある色で塗りつぶされている.
			10RPと10G,5Yと5PB等が互いに色相の角度が\mathdisplaystyle{\pi}だけが異なる色の関係になっている.
			それぞれ概ね,赤色と緑色,黄色と青色であり,これら4色が色相を4分割している.
			この考え方はHeringの反対色説と呼ばれる.
			\mathdisplaystyle{XYZ}等との座標変換は定義されていない.

			例えば,「国旗及び国歌に関する法律」(1999)によると日章旗の日章部分は紅色であるが,これをJIS慣用色名と解釈すると,日章部分はMunsell色空間の表記で3R 4/14となる.
			3Rは色相,4は明度,14は彩度である.
		\section{NCS}
			本論文でも用いているNCSについて記述する.
			NCSはHeringの反対色説を採用しつつ赤色と緑色,黄色と青色,黒色と白色を基本の色とし,クロマチックス(色の純度のような概念),黒色,白色の和が100になるように混色される.
			NCSも円柱座標系になっている(図\Reference{fig:NCS}参照).
			\begin{figure}[tbp]
				\centering
				\begin{tabular}{c}
					\begin{minipage}{\linewidth}
						\centering
						\includegraphics[width=5.5truecm]{D:/a/figs/img_page5_03.jpg}
					\end{minipage} \\
					\begin{minipage}{0.06\linewidth}
						\vspace{10truemm}
					\end{minipage} \\
					\begin{minipage}{\linewidth}
						\centering
						\includegraphics[width=5.5truecm]{D:/a/figs/img_page5_06.jpg}
					\end{minipage}
				\end{tabular}
				\captiondot{上:NCSの色相環,下:NCSの断面(R部分のみ).DICカラーデザイン\protect\cite{DIC2015}より引用}\label{fig:NCS}
			\end{figure}
	\chapter{関連研究}
		HuangらはBrattelらのアルゴリズムを基に,画像から1型2色覚者,2型2色覚者でも識別しやすい色の組み合わせを導出して色変換する計算手法を提案している\cite{Huang2007}.
		大まかな考え方は「変換の自然さ」を維持しながら色差を保つ色変換を最適化で見つけることである.
		「変換の自然さ」を維持する為に,(1)変換前後で輝度を変えない,(2)変換前で同じ色相を持つ2色は変換後も同じ色相を持つ,(3)変換前で同じ彩度を持つ2色は変換後も同じ彩度を持つように,CIELAB色空間上の\mathdisplaystyle{a^\ast b^\ast}平面を回転する.
		その回転角は,変換前の色差と合成変換(\mathdisplaystyle{\textrm{Brattel}\circ\textrm{Huang}})後の色差の差の平方和と変換前後での色差の平方和の非負定数倍を足し,これを誤差関数として最小化することで導出する.
		赤色の花や草木が写った写真に対して手法を適用した結果,変換前を2色覚者が見たら全体的にくすんだ黄色に見えていたが,変換後は花がマゼンタ色に変化し,2色覚者には青紫色に見えるようになり,花と草木を識別できるようになった.
		当研究は写真や既成のポスター等の掲示物について元々の印象をあまり損なわずに1型2色覚,2型2色覚への対応が為されるようにリデザインする際には有効であると考えられ,カラーユニバーサルデザインを推進する上で役立つと言える.
		しかし,実際には多少色が変わる為,現実を色で描写する際に3色覚者に誤解を招く可能性がある.
		また,誤差関数の第2項の定数係数はハイパーパラメータであり,この設定によって特に3色覚での色の見え方が大きく変化し得ることから,実用時は3色覚者,2色覚者(もしくは色覚シミュレータ)が共同で出力を見ながらパラメータを微調整する必要性が生じてしまう.
		本論文では,光源の設定という最小限の仮定からSVM+Bayes最適化とCIELUV色空間上の色差の代数計算,解析計算のみで色を分類する為,一度分類作業が完了すれば使い回しが可能で,零から掲示物をデザインする時にも使える.

		FuntとZhuは2型2色覚の色の見え方をBrattelらのアルゴリズムに依存しない形で計算する手法を検討している\cite{Funt2018}.
		24 bit s\mathdisplaystyle{RGB}の画像をCIE1931\mathdisplaystyle{XYZ}色空間に座標変換し,Hunt--Pointer--Estévez(HPE)変換行列で\mathdisplaystyle{LMS}色空間に変換する.
		更に\mathdisplaystyle{M=0}でM錐体の応答を無くして逆変換でs\mathdisplaystyle{RGB}画像に戻す.
		以上の変換前後の対を畳み込みニューラルネットワーク(convolutional neural network)の教師データにする.
		複数の写真に適用した結果,変換前後で大きな変化が無く,赤色や緑色も残ってしまった.
		2型2色覚が単純な\mathdisplaystyle{M=0}平面上での見え方をしていないという結論が得られている.
		本論文の2色覚変換にはBrattelらのアルゴリズムを採用していて,CIELUV色空間上の色差で分類精度の評価を行っている為,本来見えない筈の色が見える現象に翻弄されにくくなっている.
	\chapter{実験}
		\section{データセットの説明}
			本論文で使用するデータセットはNCS 1948色のAdobe\mathdisplaystyle{RGB}座標(整数値への四捨五入前,但し,計算値が負になった物は0にしてある),及び,1型2色覚,2型2色覚それぞれについて各色を43クラスでラベリングした結果である.
			データセットの概要を表\Reference{tab:dataset1},表\Reference{tab:dataset2}に示す.
			以降,列名に表れる「P」は1型2色覚,「D」は2型2色覚のことを意味するものとする.
			\tableinput{l|lrrrrrr}{ & NCS代表色 & Adobe\(R\) & Adobe\(G\) & Adobe\(B\) & \(\#\textrm{P}\) & \(\#\textrm{D}\) & \(\#\parentheses{\textrm{P}\cap\textrm{D}}\) \\ \hline
				Wt & S0500-N & 235.6629 & 234.7774 & 231.2127 & 23 & 20 & 12 \\
				plGy & S2000-N & 195.6736 & 194.7891 & 191.3287 & 40 & 36 & 20 \\
				ltGy & S3500-N & 163.9940 & 162.9809 & 160.8466 & 65 & 58 & 31 \\
				mdGy & S5000-N & 134.1809 & 134.1703 & 134.2813 & 70 & 61 & 32 \\
				dkGy & S7500-N & 86.9541 & 86.8106 & 86.6477 & 70 & 65 & 28 \\
				Bk & S9000-N & 47.2956 & 47.8310 & 49.5712 & 13 & 11 & 9 \\
				y-Wt & S0507-Y & 239.5724 & 235.2736 & 209.1030 & 74 & 70 & 63 \\
				y-plGy & S2005-Y & 197.0129 & 194.0007 & 177.4013 & 94 & 102 & 75 \\
				y-ltGy & S2502-Y & 184.4848 & 183.2185 & 174.1260 & 77 & 73 & 44 \\
				y-mdGy & S5010-G90Y & 133.8320 & 129.6478 & 109.4358 & 91 & 74 & 41 \\
				y-dkGy & S8010-G90Y & 72.0204 & 68.0469 & 54.3256 & 75 & 58 & 31 \\
				vp-Y & S1020-Y & 225.2654 & 214.4556 & 163.4146 & 64 & 62 & 53 \\
				lg-Y & S2020-Y & 196.4906 & 186.1061 & 141.9709 & 76 & 85 & 54 \\
				mg-Y & S4020-Y & 150.4791 & 138.6121 & 101.8951 & 103 & 104 & 55 \\
				dg-Y & S6020-Y & 111.1297 & 99.3336 & 69.0294 & 80 & 82 & 39 \\
				vd-Y & S7020-G90Y & 82.5625 & 75.5720 & 51.5917 & 17 & 13 & 5 \\
				pl-Y & S0540-Y & 243.6672 & 225.0307 & 135.9848 & 31 & 29 & 23 \\
				sf-Y & S2040-Y & 198.0241 & 178.9121 & 103.2851 & 73 & 78 & 46 \\
				dl-Y & S3560-Y & 150.4791 & 138.6121 & 101.8951 & 126 & 143 & 83 \\
				dk-Y & S5040-Y & 122.8045 & 104.0727 & 52.9174 & 52 & 51 & 25 \\
				lt-Y & S0560-Y & 238.1361 & 209.5451 & 87.7278 & 26 & 23 & 20 \\
				st-Y & S1080-Y & 214.4635 & 182.5300 & 0.0000 & 47 & 53 & 32 \\
				dp-Y & S2070-Y & 185.2438 & 157.4583 & 42.1169 & 30 & 39 & 15 \\
				vv-Y & S0580-Y & 238.6570 & 204.8486 & 0.0000 & 1 & 3 & 1 \\
				b-Wt & S0505-R90B & 224.6986 & 229.3882 & 233.0692 & 24 & 21 & 16 \\
				b-plGy & S0907-R90B & 211.9650 & 218.4358 & 224.1348 & 45 & 46 & 23 \\
				b-ltGy & S3005-R80B & 164.4555 & 168.6782 & 174.3469 & 34 & 35 & 12 \\
				b-mdGy & S6005-R80B & 103.6745 & 107.1785 & 114.0519 & 44 & 52 & 15}{データセットの概要}{dataset1}
			\tableinput{l|lrrrrrr}{ & NCS代表色 & Adobe\(R\) & Adobe\(G\) & Adobe\(B\) & \(\#\textrm{P}\) & \(\#\textrm{D}\) & \(\#\parentheses{\text{P}\cap\text{D}}\) \\ \hline
				b-dkGy & S7010-R90B & 73.8830 & 81.1664 & 91.7979 & 31 & 35 & 9 \\
				b-Bk & S8010-R50B & 57.8759 & 50.8047 & 62.8086 & 3 & 1 & 1 \\
				vp-pB & S0515-R90B & 204.6848 & 217.5930 & 230.3978 & 25 & 29 & 18 \\
				lg-pB & S1515-R90B & 179.8229 & 192.8497 & 206.1419 & 38 & 36 & 16 \\
				mg-pB & S4010-R90B & 132.2837 & 140.6962 & 150.5120 & 35 & 37 & 12 \\
				dg-pB & S5020-B & 93.2631 & 110.9807 & 127.1168 & 29 & 34 & 10 \\
				vd-pB & S8010-R90B & 47.3202 & 54.5546 & 66.1253 & 12 & 9 & 5 \\
				pl-pB & S0530-R90B & 174.2376 & 200.7052 & 226.8996 & 21 & 24 & 17 \\
				sf-pB & S2030-R90B & 140.7151 & 164.1106 & 190.6701 & 36 & 40 & 18 \\
				dl-pB & S4030-R90B & 100.1698 & 120.9944 & 149.8754 & 60 & 67 & 40 \\
				dk-pB & S6030-R90B & 50.6591 & 70.7194 & 97.8596 & 17 & 16 & 8 \\
				lt-pB & S1050-R90B & 122.1359 & 165.7693 & 217.2947 & 20 & 19 & 15 \\
				st-pB & S4040-R90B & 78.4195 & 105.1236 & 145.6135 & 16 & 15 & 6 \\
				dp-pB & S4550-R90B & 25.5495 & 68.4164 & 115.8239 & 28 & 24 & 21 \\
				vv-pB & S3060-R90B & 48.3133 & 96.8561 & 157.3008 & 12 & 15 & 8 \\ \hline
				合計 &  &  &  &  & 1948 & 1948 & 1107}{データセットの概要\zwspace 続き}{dataset2}
			NCSはいずれも物体色である為,光源色であるAdobe\mathdisplaystyle{RGB}色空間への座標変換の前提として光源と\mathdisplaystyle{XYZ}色空間の仮定が要求される.
			本データセットでは,白色光源にCIE標準光源D65を,\mathdisplaystyle{XYZ}色空間にCIE1931\mathdisplaystyle{XYZ}色空間(2\textdegree 視野)\cite[pp.~28--30]{Yaguchi2017a}を採用している.
			城戸らの研究(\cite[図1]{Kido2017},\cite[図1]{Kido2018})に倣って各代表色を黄青―明度平面に置くと図\Reference{fig:YB}のようになる\footnote{本来のクラス数は44であるが,y-Bkに属する色がNCSに無かったので,表\Reference{tab:dataset1},表\Reference{tab:dataset2}ではそれを無視し,図\Reference{fig:YB}ではそれを\#000000で塗っている.}.
			周囲の数字はMunsell色空間上の5Y-5PB平面における彩度,明度に対応する.
			\figureinput{width=\linewidth}{D:/a/figs/YB.png}{黄青―明度平面}{YB}
		\section{実験1}
			\subsection{手続き}
				実験1では,本データセットを1型2色覚,2型2色覚の色の見え方クラスにそれぞれ分類する分類器を作成する.
				乱数シードを基にデータセットを80/20に分割する.
				その上で,訓練データに対してSVMの分類器を導入する.
				カーネルはradial basis function kernelで固定し,\texttt{sklearn{.}svm{.}SVC}のハイパーパラメータ\mathdisplaystyle{\gamma},\mathdisplaystyle{C},\textit{class\_weight}を学習する\footnote{これ以外のパラメータはscikit-learn 0.22.1の\texttt{sklearn{.}svm{.}SVC}の初期設定のままである.}.
				ハイパーパラメータの推定にGPによるBayes最適化を用いる.
				同じ乱数シードで訓練データを更に5分割し,
				\begin{align}
					&\gamma\in\squarebrackets{2^\negativevalue{20},2^{20}}\quad\parentheses{\textrm{但し,対数一様分布(log-uniform distribution)}}\textrm{,}\label{eq:gamma}\\
					&C\in\squarebrackets{2^\negativevalue{20},2^{20}}\quad\parentheses{\textrm{但し,対数一様分布}}\textrm{,}\label{eq:C}\\
					&\textit{class\_weight}\in\braces{\textit{None},\textrm{``balanced''}}\label{eq:classweight}
				\end{align}
				という条件の下,分割された訓練データそれぞれの正解率(accuracy)の平均値を最大にするパラメータを推定する.
				チューニングに用いる\texttt{skopt{.}gp\_minimize}のパラメータについては\mathdisplaystyle{\textit{acq\_func}=\textrm{``EI''}},\mathdisplaystyle{\textit{n\_calls}=200}で設定する\footnote{これ以外のパラメータはscikit-optimize 0.5.2の\texttt{\texttt{skopt{.}gp\_minimize}}の初期設定のままである.}.
			\subsection{結果}
				分割の乱数シードは88058390である.
				実験1の結果を表\Reference{tab:result1}に纏めた.
				\tableinput{l|rr}{ & P & D \\ \hline
					\mathdisplaystyle{\gamma} & \mathdisplaystyle{1.0830\times{10}^{-6}} & \mathdisplaystyle{2.7812\times{10}^{-6}} \\
					\mathdisplaystyle{C} & \mathdisplaystyle{1.0486\times{10}^6} & \mathdisplaystyle{1.0412\times{10}^6} \\
					\textit{class\_weight} & \textit{None} & ``balanced'' \\
					正解率(訓練データ) & 0.9936 & 0.9929 \\
					正解率(テストデータ) & 0.9077 & 0.9154}{実験1の結果}{result1}

				2型2色覚は式\Equationreference{eq:gamma}--\Equationreference{eq:classweight}右辺内部で収束している.
				しかし,1型2色覚は\mathdisplaystyle{C=1.0486\times10^6\gtrsim2^{20}}より式\Equationreference{eq:C}右辺の上限に一致してしまっている.
				正解率から推測するに,1型2色覚の最適な\mathdisplaystyle{C}は\mathdisplaystyle{2^{20}}よりもやや大きいと考えられる.

				テストデータにおける混同行列(confusion matrix)を図\Reference{fig:confusion}に示す.
				上段は各々の絶対数,下段は正解クラス別に割合を取った物である.
				\begin{figure}[tbp]
					\centering
					\begin{tabular}{c}
						\begin{minipage}{0.5\linewidth}
							\centering
							\includegraphics[width=7.5truecm]{D:/a/figs/Pconmat.png}
						\end{minipage}
						\begin{minipage}{0.5\linewidth}
							\centering
							\includegraphics[width=7.5truecm]{D:/a/figs/Dconmat.png}
						\end{minipage} \\
						\begin{minipage}{0.06\linewidth}
							\vspace{10truemm}
						\end{minipage} \\
						\begin{minipage}{0.5\linewidth}
							\centering
							\includegraphics[width=7.5truecm]{D:/a/figs/Pconmat2.png}
						\end{minipage}
						\begin{minipage}{0.5\linewidth}
							\centering
							\includegraphics[width=7.5truecm]{D:/a/figs/Dconmat2.png}
						\end{minipage}
					\end{tabular}
					\captiondot{左上:1型2色覚の混同行列,右上:2型2色覚の混同行列,左下:1型2色覚の混同行列の正解クラス別割合,右下:2型2色覚の混同行列の正解クラス別割合}\label{fig:confusion}
				\end{figure}
				下段で対角成分から左または右に4列離れた成分が仄明るく光っている様子が見られる.
				これは,図\Reference{fig:YB}で明度が等しく彩度方向で隣接しているクラス同士の関係であり,Adobe\mathdisplaystyle{RGB}色空間とMunsell色空間という異なる空間ではあるものの,この関係にあるクラスが近くにあった為,誤分類が生じたと考えられる.

				訓練データでの正解率と混同行列を踏まえると,1型2色覚,2型2色覚のいずれもSVM+Bayes最適化で十分学習できたと言えるであろう.
		\section{実験2}
			\subsection{手続き}
				実験2では,1948色のデータセット全てと実験1で得たパラメータを用いて,24 bit Adobe\mathdisplaystyle{RGB}色空間上の非負格子点(\mathdisplaystyle{2^{24}}色)全てにおける1型2色覚,2型2色覚での色の見え方を予測する.
				各色は正解ラベルを2つずつ持っているはずであるが我々には未知であり,予測の精度を検証できない.
				しかし,実験1の結果から1型2色覚,2型2色覚いずれにおいても1割程度の誤分類が発生することは予想できる.
				そこで,均等色空間の一つであるCIELUV色空間に座標変換して議論する.
				均等色空間はCIELUVやCIELABの他にCIEDE2000やCAM02-UCS等様々ある\cite{Yaguchi2017b}が,CIELUVは変換が比較的単純である為,計算速度が求められる際には役立つ色空間である.
				一方で,CIELUVやCIELABは新しい均等色空間と比べて知覚の均一性が弱いという欠点がある\cite[p.~10]{Robertson1977}.

				まず,表\Reference{tab:dataset1},表\Reference{tab:dataset2}の各代表色を,CIE1931\mathdisplaystyle{XYZ}色空間\footnote{変換\mathdisplaystyle{\textrm{24bit Adobe}RGB\to\textrm{CIE1931}XYZ}はAdobe\textregistered のマニュアル\cite{Adobe2005}を参考にした.},\mathdisplaystyle{LMS}色空間\footnote{CIE標準光源D65下でのHPE変換行列を使用した.}を経由して1型2色覚,2型2色覚の\mathdisplaystyle{LMS}に写し,再度CIE1931\mathdisplaystyle{XYZ}を経由してCIELUVに座標変換する.
				任意の2色\mathdisplaystyle{\Diracket{c_\alpha}},\mathdisplaystyle{\Diracket{c_\beta}}間の色差\mathdisplaystyle{\greekDELTA E^{L^\ast u^\ast v^\ast}\parentheses{\Diracket{c_\alpha},\Diracket{c_\beta}}}はCIELUV上のEuclid距離で
				\begin{equation*}
					\greekDELTA E^{L^\ast u^\ast v^\ast}\parentheses{\Diracket{c_\alpha},\Diracket{c_\beta}}=\norm{\Diracket{c_\beta}^{L^\ast u^\ast v^\ast}-\Diracket{c_\alpha}^{L^\ast u^\ast v^\ast}}{2}
				\end{equation*}
				と表現できる.
				これを用いて,43色における距離行列を計算すると図\Reference{fig:distmatrix}を得る.
				\begin{figure}[tbp]
					\centering
					\begin{tabular}{c}
						\begin{minipage}{0.5\linewidth}
							\centering
							\includegraphics[width=7.5truecm]{D:/a/figs/Pmatrix.png}
						\end{minipage}
						\begin{minipage}{0.5\linewidth}
							\centering
							\includegraphics[width=7.5truecm]{D:/a/figs/Dmatrix.png}
						\end{minipage}
					\end{tabular}
					\captiondot{左:1型2色覚における代表色の距離行列,右:2型2色覚における代表色の距離行列}\label{fig:distmatrix}
				\end{figure}
				Munsell色空間(図\Reference{fig:YB})で近い位置関係にあった対はCIELUV色空間上でも比較的近くにあることが分かる.

				次に,各クラス\mathdisplaystyle{{\parentheses{C_i}}_{i=1}^{43}}について代表色を\mathdisplaystyle{{\hat{c}}_i\in C_i}とし,自分自身を除いて最も近い代表色\mathdisplaystyle{{\hat{c}}_{\tilde{j}\parentheses{i}}}との色差\mathdisplaystyle{\greekDELTA E_i^{L^\ast u^\ast v^\ast}}を距離行列から
				\begin{align*}
					&\tilde{j}\parentheses{i}\in\operatorname*{arg\,min}_{j\neq i}{\greekDELTA E^{L^\ast u^\ast v^\ast}\parentheses{{\hat{c}}_i,{\hat{c}}_j}}\textrm{,}\\
					&\greekDELTA E_i=\greekDELTA E^{L^\ast u^\ast v^\ast}\parentheses{{\hat{c}}_i,{\hat{c}}_{\tilde{j}\parentheses{i}}}
				\end{align*}
				で求め,代表色\mathdisplaystyle{{\hat{c}}_i^{L^\ast u^\ast v^\ast}}を中心とする半径が\mathdisplaystyle{\greekDELTA E_i}である開球\mathdisplaystyle{B\parentheses{{\hat{c}}_i^{L^\ast u^\ast v^\ast},\greekDELTA E_i}}と,半径がその半分である開球\mathdisplaystyle{B\parentheses{{\hat{c}}_i^{L^\ast u^\ast v^\ast},\greekDELTA E_i/2}}を考える.
				以下,便宜的に前者を弱採択域,後者を強採択域と呼ぶ(図\Reference{fig:adoption}参照).
				\begin{figure}[tbp]
					\centering
					\begin{tabular}{c}
						\begin{minipage}{\linewidth}
							\centering
							\begin{tikzpicture}
								\coordinate (O) at (0,0);
								\coordinate (A) at (4,3);
								\coordinate (B) at (4,4);
								\coordinate (C) at (3,1);
								\coordinate (D) at (5,0);
								\coordinate (E) at (5,2.5);
								\coordinate (F) at (5,4.5);
								\draw[loosely dashed] (O) -- (A) -- (B) -- cycle;
								\fill (O) circle [radius=2pt];
								\fill (A) circle [radius=2pt];
								\fill (B) circle [radius=2pt];
								\fill (C) circle [radius=2pt];
								\draw (O)node[below]{\mathdisplaystyle{\hat{c}_i}};
								\draw (A)node[below]{\mathdisplaystyle{\hat{c}_j}};
								\draw (B)node[above]{\mathdisplaystyle{\hat{c}_k}};
								\draw (C)node[right]{\mathdisplaystyle{c}};
								\draw (D)node[right]{\mathdisplaystyle{\partial B\parentheses{{\hat{c}}_i^{L^\ast u^\ast v^\ast},\greekDELTA E_i}}};
								\draw (E)node[right]{\mathdisplaystyle{\partial B\parentheses{{\hat{c}}_j^{L^\ast u^\ast v^\ast},\greekDELTA E_j}}};
								\draw (F)node[right]{\mathdisplaystyle{\partial B\parentheses{{\hat{c}}_k^{L^\ast u^\ast v^\ast},\greekDELTA E_k}}};
								\draw (-45:5) arc (-45:225:5cm);
								\draw (A) circle (1cm);
								\draw (B) circle (1cm);
							\end{tikzpicture}
						\end{minipage} \\
						\begin{minipage}{0.06\linewidth}
							\vspace{20truemm}
						\end{minipage} \\
						\begin{minipage}{\linewidth}
							\centering
							\begin{tikzpicture}
								\coordinate (O) at (0,0);
								\coordinate (A) at (4,3);
								\coordinate (B) at (4,4);
								\coordinate (C) at (3,1);
								\coordinate (D) at (2.5,0);
								\coordinate (E) at (4.5,2.5);
								\coordinate (F) at (4.5,4.5);
								\draw[loosely dashed] (O) -- (A) -- (B) -- cycle;
								\fill (O) circle [radius=2pt];
								\fill (A) circle [radius=2pt];
								\fill (B) circle [radius=2pt];
								\fill (C) circle [radius=2pt];
								\draw (O)node[below]{\mathdisplaystyle{\hat{c}_i}};
								\draw (A)node[below]{\mathdisplaystyle{\hat{c}_j}};
								\draw (B)node[above]{\mathdisplaystyle{\hat{c}_k}};
								\draw (C)node[right]{\mathdisplaystyle{c}};
								\draw (D)node[right]{\mathdisplaystyle{\partial B\parentheses{{\hat{c}}_i^{L^\ast u^\ast v^\ast},\fraction{1}{2}\greekDELTA E_i}}};
								\draw (E)node[right]{\mathdisplaystyle{\partial B\parentheses{{\hat{c}}_j^{L^\ast u^\ast v^\ast},\fraction{1}{2}\greekDELTA E_j}}};
								\draw (F)node[right]{\mathdisplaystyle{\partial B\parentheses{{\hat{c}}_k^{L^\ast u^\ast v^\ast},\fraction{1}{2}\greekDELTA E_k}}};
								\draw (-45:2.5) arc (-45:225:2.5cm);
								\draw (A) circle (0.5cm);
								\draw (B) circle (0.5cm);
							\end{tikzpicture}
						\end{minipage}
					\end{tabular}
					\captiondot{上:弱採択域のイメージ,下:強採択域のイメージ}\label{fig:adoption}
				\end{figure}
				24 bit Adobe\mathdisplaystyle{RGB}色空間上の各色について実験1で得たSVMで1型2色覚,2型2色覚についてのラベルを予測し,代表色に対して行ったのと同様に各2色覚への写像と座標変換で2色覚のCIELUV座標を求め,その予測ラベルの代表色の弱/強採択域に属していれば弱/強採択されるとする(図\Reference{fig:adoption}の色\mathdisplaystyle{c}の予測ラベルが\mathdisplaystyle{C_i}であるとすると,\mathdisplaystyle{c}は弱採択されるが強採択されない).
			\subsection{結果}
				分類結果をクラス別に集計した結果を表\Reference{tab:result21}--\Reference{tab:result26}に示す.
				なお,「\mathdisplaystyle{\textrm{P}'}」は1型2色覚でのSVM分類結果,「\mathdisplaystyle{\textrm{P}_\textrm{w}'}」はその分類クラス上で弱採択された物,「\mathdisplaystyle{\textrm{P}_\textrm{s}'}」は強採択された物を表す.
				「D」については2型2色覚で同様の意味を示す.
			\tableinput{l|rrr}{ & \(\#\textrm{P}'\) & \(\#\textrm{D}'\) & \(\#\parentheses{\textrm{P}'\cap\textrm{D}'}\) \\ \hline
				Wt & 31377 & 17583 & 2907 \\
				plGy & 28298 & 24367 & 5293 \\
				ltGy & 140108 & 242349 & 15740 \\
				mdGy & 122770 & 150539 & 20857 \\
				dkGy & 232508 & 184970 & 21614 \\
				Bk & 402750 & 340400 & 266528 \\
				y-Wt & 103936 & 59446 & 42617 \\
				y-plGy & 205679 & 122695 & 38095 \\
				y-ltGy & 196167 & 196636 & 39232 \\
				y-mdGy & 256462 & 359038 & 42390 \\
				y-dkGy & 205151 & 107913 & 24868 \\
				vp-Y & 160347 & 122127 & 68861 \\
				lg-Y & 334141 & 381390 & 80118 \\
				mg-Y & 396539 & 688404 & 107566 \\
				dg-Y & 381723 & 375749 & 73040 \\
				vd-Y & 300874 & 202228 & 72899 \\
				pl-Y & 359530 & 202534 & 149854 \\
				sf-Y & 746453 & 478222 & 167422 \\
				dl-Y & 710716 & 1274158 & 258420 \\
				dk-Y & 588299 & 1208235 & 228748 \\
				lt-Y & 1405081 & 721005 & 570017 \\
				st-Y & 425888 & 501260 & 207420 \\
				dp-Y & 1206695 & 1079479 & 376267 \\
				vv-Y & 140658 & 96129 & 29262 \\
				b-Wt & 39645 & 17363 & 2583 \\
				b-plGy & 58964 & 82203 & 6365 \\
				b-ltGy & 100481 & 194246 & 7622 \\
				b-mdGy & 175929 & 332281 & 11748}{実験2の結果(採択なし)}{result21}
			\tableinput{l|rrr}{ & \(\#\textrm{P}'\) & \(\#\textrm{D}'\) & \(\#\parentheses{\textrm{P}'\cap\textrm{D}'}\) \\ \hline
				b-dkGy & 81492 & 124207 & 15222 \\
				b-Bk & 36231 & 32428 & 5045 \\
				vp-pB & 139329 & 88204 & 21866 \\
				lg-pB & 122432 & 208697 & 18188 \\
				mg-pB & 160575 & 114763 & 20359 \\
				dg-pB & 192398 & 161755 & 27497 \\
				vd-pB & 169603 & 53683 & 35852 \\
				pl-pB & 216066 & 274333 & 124356 \\
				sf-pB & 189202 & 217637 & 60312 \\
				dl-pB & 351472 & 491051 & 132165 \\
				dk-pB & 167737 & 167473 & 95820 \\
				lt-pB & 1443214 & 1702163 & 933824 \\
				st-pB & 190276 & 423639 & 31258 \\
				dp-pB & 3218012 & 1987972 & 1955098 \\
				vv-pB & 642008 & 966262 & 224085 \\ \hline
				合計 & 16777216 & 16777216 & 6639300}{実験2の結果(採択なし)\zwspace 続き}{result22}
			\clearpage
			\tableinput{l|rrr}{ & \(\#\textrm{P}_\textrm{w}'\) & \(\#\textrm{D}_\textrm{w}'\) & \(\#\parentheses{\textrm{P}_\textrm{w}'\cap\textrm{D}_\textrm{w}'}\) \\ \hline
				Wt & 0 & 0 & 0 \\
				plGy & 0 & 0 & 0 \\
				ltGy & 0 & 0 & 0 \\
				mdGy & 0 & 0 & 0 \\
				dkGy & 0 & 0 & 0 \\
				Bk & 0 & 0 & 0 \\
				y-Wt & 0 & 0 & 0 \\
				y-plGy & 0 & 0 & 0 \\
				y-ltGy & 0 & 0 & 0 \\
				y-mdGy & 0 & 0 & 0 \\
				y-dkGy & 0 & 0 & 0 \\
				vp-Y & 0 & 0 & 0 \\
				lg-Y & 0 & 0 & 0 \\
				mg-Y & 0 & 0 & 0 \\
				dg-Y & 10197 & 31885 & 0 \\
				vd-Y & 55791 & 106926 & 22575 \\
				pl-Y & 0 & 0 & 0 \\
				sf-Y & 0 & 78 & 0 \\
				dl-Y & 43412 & 41862 & 0 \\
				dk-Y & 290604 & 458202 & 85154 \\
				lt-Y & 211248 & 49701 & 10640 \\
				st-Y & 241 & 0 & 0 \\
				dp-Y & 569421 & 638624 & 144691 \\
				vv-Y & 85561 & 35283 & 19685 \\
				b-Wt & 0 & 0 & 0 \\
				b-plGy & 0 & 0 & 0 \\
				b-ltGy & 0 & 0 & 0 \\
				b-mdGy & 0 & 0 & 0}{実験2の結果(弱採択)}{result23}
			\tableinput{l|rrr}{ & \(\#\textrm{P}_\textrm{w}'\) & \(\#\textrm{D}_\textrm{w}'\) & \(\#\parentheses{\textrm{P}_\textrm{w}'\cap\textrm{D}_\textrm{w}'}\) \\ \hline
				b-dkGy & 0 & 0 & 0 \\
				b-Bk & 0 & 0 & 0 \\
				vp-pB & 0 & 0 & 0 \\
				lg-pB & 0 & 0 & 0 \\
				mg-pB & 0 & 0 & 0 \\
				dg-pB & 0 & 161563 & 0 \\
				vd-pB & 0 & 0 & 0 \\
				pl-pB & 0 & 0 & 0 \\
				sf-pB & 0 & 0 & 0 \\
				dl-pB & 0 & 0 & 0 \\
				dk-pB & 55788 & 114623 & 39233 \\
				lt-pB & 0 & 0 & 0 \\
				st-pB & 0 & 0 & 0 \\
				dp-pB & 163849 & 286178 & 117445 \\
				vv-pB & 0 & 89715 & 0 \\ \hline
				合計 & 1486112 & 2014640 & 439423}{実験2の結果(弱採択)\zwspace 続き}{result25}
			\clearpage
			\tableinput{l|rrr}{ & \(\#\textrm{P}_\textrm{s}'\) & \(\#\textrm{D}_\textrm{s}'\) & \(\#\parentheses{\textrm{P}_\textrm{s}'\cap\textrm{D}_\textrm{s}'}\) \\ \hline
				Wt & 0 & 0 & 0 \\
				plGy & 0 & 0 & 0 \\
				ltGy & 0 & 0 & 0 \\
				mdGy & 0 & 0 & 0 \\
				dkGy & 0 & 0 & 0 \\
				Bk & 0 & 0 & 0 \\
				y-Wt & 0 & 0 & 0 \\
				y-plGy & 0 & 0 & 0 \\
				y-ltGy & 0 & 0 & 0 \\
				y-mdGy & 0 & 0 & 0 \\
				y-dkGy & 0 & 0 & 0 \\
				vp-Y & 0 & 0 & 0 \\
				lg-Y & 0 & 0 & 0 \\
				mg-Y & 0 & 0 & 0 \\
				dg-Y & 0 & 0 & 0 \\
				vd-Y & 0 & 0 & 0 \\
				pl-Y & 0 & 0 & 0 \\
				sf-Y & 0 & 0 & 0 \\
				dl-Y & 0 & 94 & 0 \\
				dk-Y & 7244 & 83513 & 1118 \\
				lt-Y & 0 & 0 & 0 \\
				st-Y & 0 & 0 & 0 \\
				dp-Y & 139260 & 216711 & 6893 \\
				vv-Y & 0 & 0 & 0 \\
				b-Wt & 0 & 0 & 0 \\
				b-plGy & 0 & 0 & 0 \\
				b-ltGy & 0 & 0 & 0 \\
				b-mdGy & 0 & 0 & 0}{実験2の結果(強採択)}{result23}
			\tableinput{l|rrr}{ & \(\#\textrm{P}_\textrm{s}'\) & \(\#\textrm{D}_\textrm{s}'\) & \(\#\parentheses{\textrm{P}_\textrm{s}'\cap\textrm{D}_\textrm{s}'}\) \\ \hline
				b-dkGy & 0 & 0 & 0 \\
				b-Bk & 0 & 0 & 0 \\
				vp-pB & 0 & 0 & 0 \\
				lg-pB & 0 & 0 & 0 \\
				mg-pB & 0 & 0 & 0 \\
				dg-pB & 0 & 2718 & 0 \\
				vd-pB & 0 & 0 & 0 \\
				pl-pB & 0 & 0 & 0 \\
				sf-pB & 0 & 0 & 0 \\
				dl-pB & 0 & 0 & 0 \\
				dk-pB & 0 & 0 & 0 \\
				lt-pB & 0 & 0 & 0 \\
				st-pB & 0 & 0 & 0 \\
				dp-pB & 0 & 0 & 0 \\
				vv-pB & 0 & 0 & 0 \\ \hline
				合計 & 146504 & 303036 & 8011}{実験2の結果(強採択)\zwspace 続き}{result26}
	\chapter{結論}
	\chapter*{謝辞}
%	本研究を進めるに当たり,指導教官の池田大輔准教授からは多大な助言を賜りました.
%	研究テーマの決定,冬休み中の急激なテーマ変更のお願いに対しても親身に接し,常に温かいご指導をしてくださったことが本論文の完成に繫がりました.
%	厚く感謝を申し上げます.
%	
%	また,本学芸術工学研究院デザイン人間科学部門の須長正治准教授からは,NCSのデータセットの提供,並びに色彩学や色覚異常についてのご教授を頂きました.
%	厚く御礼申し上げます.
%
%	毎日共に研究を行った池田研究室の皆様には,常に適切な指摘によって研究を支えていただき,大変有意義な1年間の研究生活を過ごせました.
%	ありがとうございました.
%	特に同期の永溝孝太君から頂いたアドバイスや彼との他愛ない会話が大きな支えになりました.
%	ここに感謝の意を表します.
%
%	最後になりましたが,本研究を応援し,支援してくださった多くの方々に感謝いたします.
%	本当にありがとうございました.
	\clearpage
	\addcontentsline{toc}{chapter}{\refname}
	\bibliography{ref}
	\bibliographystyle{Cpjeconunsrt}
	%\bibliographystyle{jecon-no-sort}
	%\bibliographystyle{junsrt}
	%\bibliographystyle{IEEEtranS}
\end{document}